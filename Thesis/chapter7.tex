\chapter{Conclusion}

The developed application succeeds in offering a way to incorporate exercises with playing games. A physical therapist can come up with exercises that fit the needs of the patient and link each of these exercises to a keyboard key. As such, any game that can be played with keyboard controls can be played with this application, independent of the chosen exercises. This gives patients a very large pool of games to choose from and helps keeping them motivated to do all needed exercises.\\

The therapist can train all required exercises without any programming knowledge. Machine learning algorithms are used to evaluate which of the exercises is performed by the patient and allow the therapist to support the patient in a different way, by focusing together with the patient on playing the game instead of counting the number of times an exercise is done. The way SVM is used to recognize the executed exercise minimizes the time between a gesture execution and a reaction from the application. This allows for a more pleasant experience for the patient while playing games, as they feel more responsive.\\

Special attention is paid to the application running smooth during its use. The application updates the user's avatar drawn on the interface at a rate of 30 fps, which is as fast as the Kinect 2.0 camera can measure new data. As an important form of feedback, this helps mimicking user movement in real-time and keeps the user immersed into the experience.\\

The GUI uses visual elements that express their function through feedforward. The goal is to use objects that users can recognize from real life and inherently know how to interact with. A gentle learning curve paves the way for new users in minimizing the amount of additional information needed before the application can be used. By naturally coupling action to reaction, feedback informs the user of how an interaction affects the application and makes it easier to understand the result of it.\\
