\chapter{Conclusion}

The developed application, called OmniPlay, succeeds in offering a way to incorporate exercises with playing games. A physical therapist can come up with gestures that fit the needs of the patient and link each of these gestures to a keyboard key. As such, any game that can be played with keyboard controls can be played with this application, independent of the chosen gestures. This makes OmniPlay possibly more profitable than custom designed games because it gives patients a very large pool of games to choose from and helps keeping them motivated to do all needed gestures.\\

The therapist is able to input all required gestures without any programming knowledge. Machine learning algorithms are used to evaluate which of the gestures is performed by the patient and allow the therapist to support the patient in a different way, by focusing together with the patient on playing the game instead of counting the number of times a gesture is done. The way SVM is used to recognize the executed gesture minimizes the time between a gesture execution and a reaction from the application. This allows for a more pleasant experience for the patient while playing games, as they feel more responsive.\\

Special attention is paid to the application running smooth during its use. The application updates the user's avatar drawn on the interface at a rate of 30 FPS, which is as fast as the Kinect 2.0 camera can measure new data. As an important form of feedback, this helps mimicking user movement in real-time and keeps the user immersed into the experience.\\

The GUI uses visual elements that express their function through feedforward. The goal is to use objects that users can recognize from real life and inherently know how to interact with. A gentle learning curve paves the way for new users in minimizing the amount of additional information needed before the application can be used. By naturally coupling action to reaction, feedback informs the user of how an interaction affects the application and makes it easier to understand the result of it.\\

\chapter{Christiaan Conclusion}

The developed application, called OmniPlay, succeeds in offering a way to incorporate exercises with playing games. A physical therapist can come up with gestures that fit the needs of the patient and link each of these gestures to a keyboard key. As such, any game that can be played with keyboard controls can be played with this application, independent of the chosen gestures. This makes OmniPlay possibly more profitable than custom designed games because it gives patients a very large pool of games to choose from and helps keeping them motivated to do all needed gestures.\\

A machine learning algorithm called support vector machines (SVM) is used to evaluate the gestures performed by the user, virtually pressing the assigned keyboard button when a recorded gesture is recognized. By employing a system of splitting the gesture into stattic postures that need to happen in a particular order, the OmniPlay can be used to effectively play games using gestures. Though the exact amount and nature of gestures was not the focus of this thesis.

Using prototypes consisting of mime or hints patterns of interaction combined with WIMP elements, user tests were conducted to come to the conclusion that a combination mild usage of the mime pattern with WIMP element is preferred while the most important aspects of the UI are: it's simplicity, that all possible action are permanently on screen and usage of clear and familiar metaphors. From these criteria, a coded prototype was developed of one screen where the user can record and evaluate gestures to be used to play any game. A user test showed that the physical therapist is able to input all required gestures despite lacking any programming knowledge. After some introductory explanation, he is able to find effective gestures on his own that he used to play an online game.  \\


Needs some more conclusion about the properties of the user interface but that is for tommorow