\chapter{Appendix A}
Explanation about the appendix.

The second paper prototype combines mime pattern elements with more standard WIMP elements to streamline the process. In \textbf{ fig 3} the first screen can be seen. On the left of the screen the screen can be exited by grabbing the orange part of the lever and pulling it downwards. A new recording can be started by grabbing the orange part of the cord and pulling downwards. The record screen as seen in \textbf{ fig 4} is now shown. The window shown is what is going to be recorded, the red one shows where the program will count down from 3 until the recording starts allowing the user for some time to get to his start position. All other UI elements are grayed out to indicate that they are not available to interact with. Once the recording is finished the program returns to the screen in \textbf{ fig 3}. When the user moves into the orange circle on the ground the program switches to a different mode as seen in \textbf{ fig 4} which is called the manage screen. The window on the right of the screen will replay the recording that is highlighted in orange in the scrollbar automatically. The user can pause this function by pushing the orange slide button above the scrollbar to the left, grabbing it is not necessary. Each block in the scrollbar is linked to a recording and contains a static previews of their recording. The recording that is highlighted orange can be pushed into the trash, grabbing it is also not necessary here. On the right side of the user is an orange scroll wheel that is connected to the scrollbar to indicate it's association with the scrollbar, the orange tint difference can be used to indicate the speed and the direction in which the wheel is turning. The idea is that the user can move an right hand over the scroll wheel to scroll through the items in the scrollbar, simultaneously the user can use his left hand to delete an item in the scrollbar. The gray block next to the scrollbar is an indication of position within the scrollbar.

\begin{figure}[H]
	\begin{center}
		\includegraphics[width=12.5cm, height=7cm]{KUL.png}
		\caption{\emph{The standard screen of the second paper prototype}}
		\label{The standard screen of the second paper prototype}
	\end{center}
\end{figure}

\begin{figure}[H]
	\begin{center}
		\includegraphics[width=12.5cm, height=7cm]{KUL.png}
		\caption{\emph{The record screen of the second paper prototype}}
		\label{The first paper prototype}
	\end{center}
\end{figure}

\begin{figure}[H]
	\begin{center}
		\includegraphics[width=12.5cm, height=7cm]{KUL.png}
		\caption{\emph{The manage screen of the second paper prototype}}
		\label{The first paper prototype}
	\end{center}
\end{figure}