\chapter{Introduction}

There are many different reasons for requiring physical therapy. They vary from recovering from a car crash to being born with physical disabilities. The goal of this therapy is respectively to make sure that the patient can recover as much as possible from his injuries or to train the muscles, preventing their situation from deteriorating.\\

Especially for children, the physical exercises performed during therapy can be demotivating. Combining these exercises with playing computer games leads to an increased willingness to continue with the exercises. However, the classic approach related to games that incorporate physical exercises is not economically sustainable.\\

The purpose of this thesis is to present a more sustainable solution to this problem in a way that offers more flexibility to the therapist, both in terms of the exercises that need to be done by the patients and the games they can control and play by performing the exercises.


\section{Discussion of the problem}

Patients often see exercises as a part of physical therapy as being fatiguing, monotonous and tedious. This problem is even more prominent with young patients and can quickly demotivate them, especially when the exercises are uncomfortable or painful to perform.\\ %SOURCE

Computer games already exist that can support physical therapies. For instance, there are games that run on Microsoft's XBox console and accept user input through the Kinect 3D camera, or Nintendo's Wii console that use a controller with accelerometers. Additionally, there are also games that can run on a computer, using any combination of sensors for user input, and are specifically made for use by physical therapists and their patients. However, all these games have in common that they are very static by nature and are not often suited for the specific needs of a patient. For instance, a game that focuses on arm movement is not very effective at exercising the leg muscles of a patient. In other words, concerning the therapy, the game is meaningless for a patient that had leg surgery.\\

In addition, these static games have fixed input gestures the patients have to perform in order to control the action on screen. Even if the gestures perfectly fit the exercise requirements for a specific patient, the decrease in attractiveness of playing the same game over and over again is problematic and loses its long-term effect. Physical therapist Dries Lamberts has experience with children with disabilities and uses computer games as a form of exercise. The initial reaction of the children to these games is positive. They are more engaged in doing physical exercises and feel motivated while doing so. On the downside, this effect only lasts until the novelty of the game wears off. After that, the children lose interest in the game and, as such, in performing the exercises.\\

As games are expensive to develop, and motion-based games in particular, there is no wide variety of games that are tailor made for people with specific needs and at the same time offer varied gameplay mechanics to remain interesting over long periods of time. On the other hand, producing these kind of games is not economically sustainable. From the patient's point of view, the right type of exercises need to be supported by the game and the game itself has to be considered fun as well by the patients for them to keep invested in it.


\section{Purpose of the research}

This thesis focuses on an application that allows for a more flexible and dynamic solution to the above discussed problem. It lets the physical therapist choose what exercises the patient has to do and how these can be used as an input to control any computer game of choice. All of this can be done without requiring the therapist to have any programming knowledge.\\

The goal is to have an application the therapist can control mainly using the Kinect camera. It needs to be both simple and efficient to do, in order to minimize the setup time before a patient can start playing. It is necessary to research the type of interface that is required to meet these objectives, in addition to finding out the easiest way to interact with an on-screen application using a camera for input.