\chapter*{Abstract}

Using games as part of physical therapy improves patient motivation to perform the necessary exercises. The developed application allows physical therapists to choose exercises that fit the needs of the patients and record them using the Kinect 2.0 camera. When the patient executes an exercise, algorithms relying on support vector machine (SVM) are used to recognize the exercise. Keyboard keys can be linked to an exercise. When an exercise is performed, the linked key is pressed. This makes it possible to interact with any available game relying on key inputs.\\

Research is conducted on the subject of human-computer interaction (HCI) involving the use of the Kinect camera to interact with the application in what is called a natural user interface (NUI). Several prototypes are designed that focus on different interaction principles and are for the purpose of this thesis divided into two categories: the mime pattern and the music conductor pattern. The chosen implemented prototype of the interface allows users to interact with the menus and items appearing on-screen without the use of traditional buttons. Interactive elements like pulling ropes are present in the interface, providing both feedforward and feedback to the user in an effort to simplify the interaction process.\\

As part of the user-centered design approach, the interface is evaluated by conducting a user test with a physical therapist. Conclusions from this test are that some elements take more time to understand how it is possible to interact with. With a short explanation, it is possible to minimize the time initially required to achieve this. The application as a whole is simple to use after performing the process of recording gestures once.\\

\underline{Key words}: games, gesture recognition, Kinect 2.0, physical therapy, support vector machine, user interface


\chapter*{Extended abstract}

Bij kinesitherapie is het vaak vermoeiend en eentonig voor jongere pati\"enten om dezelfde oefeningen steeds te herhalen. omputerspellen die toelaten om te interageren met het spel door middel van bewegingen. kunnen deel uitmaken van bijvoorbeeld revalidatietherapie of 