\chapter*{Abstract}

%Using games as part of physical therapy improves patient motivation to perform the necessary exercises. The developed application allows physical therapists to choose exercises that fit the needs of the patients and record them using the Kinect 2.0 camera. When the patient executes an exercise, machine learning algorithms are used to recognize the exercise, which activates the keyboard key linked to that exercise. This makes it possible to interact with any available game relying on key inputs.\\

%Research is conducted on the subject of human-computer interaction (HCI) involving the use of the Kinect camera to interact with the application in what is called a natural user interface (NUI). Several prototypes are designed that focus on different interaction principles and are divided into two categories: the mime pattern and the hints pattern. The chosen implemented prototype of the interface allows users to interact with the menus and items appearing on-screen without the use of traditional buttons. Interactive elements like pulling ropes are present in the interface, providing both feedforward and feedback to the user in an effort to simplify the interaction process.\\

%As part of the user-centered design approach, the interface is evaluated by conducting a user test with a physical therapist. Feedback is obtained both during the prototyping phase, as well as after having implemented the prototype. A conclusion from this test is that it takes more time to understand how some elements can be interacted with. With a short explanation, it is possible to decrease the time required to achieve this. The application as a whole has a low learning curve as the focus on simplicity results in the user being familiar with all required actions after recording an exercise once.\\

%\underline{alternative abstract}

Patients are often demotivated by pain or boredom during physical therapy and previous attempts at using motion controlled games to improve this worked, but had limited exercises and monotonous game play that proved to make it economically inviable. In this thesis a more versatile program is developed by using the kinect 2.0 3D-camera to allow a physical therapist to train a support vector machine (SVM) algorithm to recognize any exercise and link it to an action in any game that uses keyboard input. Though this thesis is more focused on the human-computer interaction (HCI) aspect of designing a user-friendly motion controlled natural user interface (NUI) for this program.\\

For the development of the NUI several special paper-prototypes are designed that focus on different interaction principles. After user feedback a principle was chosen and further developed into a coded prototype that uses straight-forward interaction with good feedforward and feedback in an effort to streamline the interaction process. After a final user test, small changes to the prototype were applied. For the recognition of exercises a few choices had to be made about the processing of the input data.\\


The final conclusion is that most users recognize the required interactions fairly quickly and are able to efficiently control it rapidly.Though there were some instances were the user expected a different reaction after an action. It was obvious that the application will need some additional information to teach the user how to record good exercices. During recognition several positions or movements can be distinguished with good accuracy. \\


\underline{Key words}: games, gesture recognition, Kinect 2.0, physical therapy, support vector machine, user interface


\chapter*{Extended abstract}

Bij kinesitherapie is het vaak vermoeiend en eentonig voor jongere pati\"enten om dezelfde oefeningen steeds te herhalen. omputerspellen die toelaten om te interageren met het spel door middel van bewegingen. kunnen deel uitmaken van bijvoorbeeld revalidatietherapie of 