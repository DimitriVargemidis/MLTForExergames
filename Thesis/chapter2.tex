\chapter{Literature study}

\section{Human-computer interaction}

Gesture-based interfaces gain popularity due to the advancing technology of sensors and processors \citep*{Jacob2008}. Because of this, technologies like the Kinect camera become more affordable and accessible for individuals. While interfaces supporting gesture input are referred to as natural user interfaces (NUI), this is more of a marketing term than an actual scientific designation. Natural interfaces don't feel natural to interact with just because they rely on gesture-based input. In fact, feedback and feedforward is limited to what is shown on-screen, but is an essential part of the interaction \citep*{Wensveen2004}. The user does not really hold the object shown on the screen to be able to interact with it naturally, so the feedback a user can get is limited. \citep*{Norman2010}

Notwithstanding this limitation, gestures are powerful tools for human-computer interaction (HCI). However, gestures as an input offer a wide variety of options and with increasing popularity, a wider variation of interfaces and gestures to interact with them exist. This is problematic as users need to get familiar to each gesture-based It's full potential can only be achieved when some kind of standard is developed and widely used. 

\begin{itemize}

\item Limit setup time for the therapist
\item simplify the process
\item provide feedback for the therapist
\item Papers/artikels over framework, natuurlijke interfaces,\ldots
\item \ldots
\end{itemize}


\section{Similar research}

\begin{itemize}
\item Vergelijking van opzet met andere oplossingen 
\item Te trekken lessen uit deze oplossingen (vb: feedback voor de kinesist is belangrijk, opstellen van het systeem mag niet veel tijd in beslag nemen,\ldots)
\item \ldots
\end{itemize}