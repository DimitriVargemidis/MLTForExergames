\chapter{Related work}
\label{chapter: related work}

The literature study focuses on the aspects of human-computer interaction (HCI) related to gesture-based input and natural interfaces. This is followed by a closer look on research done that is directly related to using exergames as part of physical therapy.

\section{Human-computer interaction}


Gesture-based interfaces gain popularity due to the advancing technology of sensors and processors \cite{Jacob2008}. Because of this, technologies like the Kinect camera become more affordable and accessible for individuals. While interfaces supporting gesture input are referred to as natural user interfaces (NUI), this is more of a marketing term than an actual scientific designation. Natural interfaces don't feel natural to interact with just because they rely on gesture-based input. The user does not really hold the object shown on the screen to be able to interact with it naturally, so the feedback a user can get is limited \cite{Norman2010}. Direct manipulation of objects that interact with an application offer more natural feedback \cite{Shneiderman2010}. The downside is that additional objects are required for interacting with the application.\\

Notwithstanding the limited feedback when not using additional objects, gestures are powerful tools for human-computer interaction. However, gestures as an input offer a wide variety of options and with increasing popularity, a wider variation of interfaces and gestures to interact with them emerge. This is problematic as users need to get familiar with a gesture-based interface each time they encounter a new one. The full potential of this kind of interfaces can only be achieved when some kind of standard is developed and widely used \cite{Norman2010}. Frameworks based on reality-based interaction can be used to unify different research areas concerning interface design and to compare these designs to one another \cite{Jacob2008}.\\

When designing an interface and ways to interact with them, it is essential to make clear what actions are required through feedforward and what the results of those actions are through feedback. Because of that, feedback and feedforward are powerful tools. Designing a gesture-based interface comes down to trying to have natural coupling between action and reaction. Natural coupling can be evaluated by six aspects: time, location, direction, dynamics, modality and expression.\cite{Wensveen2004}\\

The time aspect requires that there is no perceivable delay between the execution of an action and the result shown on the screen. The direction and location of action and reaction have to match as well. The modality is tightly coupled to the dynamics and direction, as it must be possible to visually perceive the result of an action and the visual representation of the reaction must be similar to the action in terms of dynamics, meaning that the speed and acceleration of both have to match. The aspect of expression is also related to the dynamics of the action. The interface has to visually represent to way the user performed an action. If the action is done quickly and not very accurately, it must be reflected by the interface. If feedback as well as feedforward are able to support these six aspects, the user perceives the interface as more intuitive to interact with and action and function become more closely related. The more aspects are fulfilled, the higher the natural coupling is. \cite{Wensveen2004}\\


\section{Exergames as part of physical therapy}

Using games as part of physical therapy has been the subject of many research papers \cite{Geurts2011}\cite{Hondori2014}\cite{Lange2012}\cite{Chang2011}. Since it is found that exergames improve motivation \cite{Brauner2013} and offer physical, social and cognitive benefits, they are more frequently used by physical therapists \cite{Peng2011}\cite{Staiano2011}.\\

While the reliance of the patient on the therapist can be reduced by using exergames, the rate of rehabilitation does not increase \cite{Dahl2014}. Because of a decreased reliance on the therapist, he can focus on supporting the patient during play \cite{Annema2013}. Since these games can be a tool of aid to the therapists, it should not hinder them in any way, nor increase the amount of work or effort they put into the preparation of a therapeutic session. For this reason, setting up everything needed before starting to play should not take up too much time. This includes but is not limited to: the time to set up the used sensors, the time the application takes to start up and the time is needed for the application to process data.\\

Within the same mindset of make live easier for physical therapists, the application can be a useful source of information for the therapist, as it can provide him with details of the progress his patient is making either during one session or over the course of long-term therapy. Again, this allows the therapist to focus on the essence of the therapy instead of on taking notes, keeping close track of how the patient is doing \cite{Annema2013}.\\