Designing and implementing a user interface utilizing the relatively new Kinect camera is a challenging, albeit fascinating subject. We encountered many problems along the way, for which we enjoyed coming up with creative solutions. In the process, we were supported by many people we definitely would like to thank.\\

This thesis would not have been possible to write without our supervisor professor Luc Geurts and co-supervisor Karen Vanderloock from e-Media lab, who provided us with this subject, invested their time and shared their knowledge and experience with us, allowing us to come up with solutions that have lead to a fully functional result.\\

During the year, we've had several occasions to discuss the subject of machine learning and support vector machine (SVM) with our fellow students Victor Martens and Alexander Kerckhofs, who helped with making us aware of the pitfalls and shortcomings of the used techniques and joined us to discuss the requirements and come up with a creative solution concerning the use of SVM.\\

User testing is very important when designing an application with a specific audience in mind. As such, we are grateful for the useful feedback of physical therapist Dries Lamberts of Windekind, who took the time to meet us twice and evaluate both our prototype as well as the final result. But even more importantly, he shared with us his passion for taking care of children with disabilities, their need for physical exercise and the importance of making sure this is both meaningful and entertaining.\\

Last but not least, we would also like to thank our parents, family and friends, not only for their unwavering support during the course of past year while working on this master's thesis, but also for trying out our application and giving valuable feedback.